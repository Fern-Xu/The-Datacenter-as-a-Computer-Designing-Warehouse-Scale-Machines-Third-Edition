\documentclass[UTF8]{ctexart}
\usepackage{indentfirst} 
\title{数据中心与仓储级计算机设计}
\author{徐秉正}
\date{\today}
\begin{document}
\maketitle
\tableofcontents

\section{工作负载与软件基础设施}

\section{WSC硬件构成}
\subsection{服务器}
	如前文所述,WSC的结构很大程度上取决于其硬件组成,这有点像为微处理器选择逻辑元件
	或是为服务器平台选择正确的芯片组与器件。在WSC领域,我们主要关注服务器硬件、网络
	结构与层次化的储存器件。本章将介绍这些器件并帮助选型。
	\subsubsection{大型SMP通信效率的影响}
	有许多因素使得中档服务器集群成为当前WSC的首选,其中最主要的原因是与早期为了高性能
	计算与科学计算搭建的顶级共享存储系统相比,中档服务器集群拥有更高的性价比。CPU核心
	数目持续不断的增长使得大部分VM(虚拟机)或任务可以在双插槽服务器上流畅运行。这样的
	服务器平台的许多核心部件与个人计算机通用,在规模经济中获益颇多。
	\subsubsection{高性能服务器与低性能服务器对比}

\subsection{计算加速器}
	\subsubsection{GPU}
	\subsubsection{TPU}

\subsection{网络}
	\subsubsection{集群网络}
	\subsubsection{主机网络}

\subsection{存储}
	\subsubsection{硬盘托盘与无盘服务器}
	\subsubsection{非结构化WSC储存}
	\subsubsection{结构化WSC储存}
	\subsubsection{GPU}

\subsection{设计的权衡}
	\subsubsection{系统平衡:层次化存储}
	\subsubsection{量化延迟、带宽与容量}


\section{数据中心基础:建筑、电力与冷却}

\section{能源与效率}

\section{对成本建模}

\section{故障处理及维修}

\section{结束语}
\end{document}