\documentclass[UTF8]{ctexart}
\usepackage{indentfirst} 
\usepackage{enumitem}
\title{Note}
\author{徐秉正}
\date{\today}
\begin{document}
\maketitle
\tableofcontents
\section{一些基础知识和概念}
\subsection{DLP TLP ILP RLP}
\begin{description}
    \item[]Classes of application parallelism
    \begin{description}
        \setdescription{labelsep=\textwidth}
    \item[DLP]Data-level parallelism 数据级并行,同时操作许多数据项
    \item[TLP]Thread-level parallelism 并行处理单独任务
    \end{description}
    \item[]实现方式
    \begin{description}
    \item[ILP]Instruction-Level Parallelism 多指令流多数据流
    \item[RLP]Request-level parallelism 
    \end{description}
\end{description}
\subsection{SISD SIMD MIMD}
\begin{description}
    \item[SISD]Single Instruction Single Data 单指令流单数据流
    \item[SIMD]Single Instruction Multiple Data 单指令流多数据流
    \item[MIMD]Multiple Instruction Multiple Data 多指令流多数据流
\end{description}
\subsection{RISC}
\section{Datacenter/WSC/HPC Cluster/Server}
显然Cluster和WSC都是请求级并行的MIMD体系结构
\end{document}